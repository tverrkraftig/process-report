\chapter{Teori}
For å studere prosessen gruppa gikk gjennom i EiT er det viktig med et godt teoretisk fundament. 
Teorien som ble hentet fra boka \textit{Arbeid i Team} av Levin \& Rolfsen, men deler er også hentet fra kompendiumet i faget og videoforelesninger som er tilgjengelig på nett. 
Kompendiumet er satt sammen av fire forsjellige bøker:
\begin{enumerate}
    \item Wheelan, Susan A. (2009): Creating Effective Teams. A Guide for Members and Leaders.
    \item Johnson, David \& Frank (2013): Joining together. Group theory and group skills.
    \item Schwarz, Roger (2002): The skilled facilitator. 
    \item Hjertø, Kjell B. (2013): TEAM.
\end{enumerate}

\section{Gruppeteori}
Det finnes mange forskjellige definisjoner på en arbeidsgruppe/team. 
Definisjonen til Katzenbach og Smith lyder som følger:

\begin{center}
\textit{Et team defineres som flere personer som arbeider sammen for å oppnå et felles mål. 
Medlemmene må være avhengige av hverandre på en eller annen måte.}
\newline 
(Katzenbach \& Smith, 1993)
\end{center}

Gruppe settes med andre ord ofte sammen for å oppnå et mål som er vanskelig/umulig å oppnå alene. 
Det er ikke slik at det alltid er fordelmessig å jobbe i gruppe, men komplekse flerfaglige/uoversiktelige oppgaver blir gjerne enklere å løse i grupper.
Gruppearbeid kan også gi løsninger preget av en større grav av nyskapning og kreativitet. 
Generelt er den største motivasjonen for å arbeide i en gruppe at større prosjekter har en veldig stor grad av tverrfaglighet. 
Dette er noe som kommer godt frem i arbeidslivet også. 
\vspace{\secspace}

En \textbf{homogen gruppe} vil si at gruppemedlemmene har liten faglig spredning. 
En slik gruppe vil være mest effektiv når prosjektet har begrenset omfang og er lite flerfaglig. 
Fordelen med en slik gruppe er at gruppemedlemmene kommer ofte godt overens og har derfor gjerne mindre sosiale utfordringer. 
Det er også noen bakdeler med en homogen gruppe. 
Homogene grupper har en tendens til å unngå risikoer, og dermed går glipp av muligheter. 
De har også problemer med å tilpasse seg dynamiske situasjoner.
\vspace{\secspace}

Det motsatte av en homogen gruppe er en \textbf{heterogen gruppe}. 
Dette er en gruppe preget faglig spredning og er ofte bedre til å ta viktige beslutninger samt tenke nytt og kreativt enn homogene grupper. 
Til tross for dette er det ikke gitt at en faglig spredt gruppe resulterer i suksess. 
Ofte kan heterogene grupper ha større utfordringer når det kommer til intern kommunikasjon eller problemer med at medlemmer blir deffansive og derfor mindre produktive. 

\section{Gruppeutvikling}
Det har vært gjort utallige forsøk på å beskrive hvordan en gruppe utvikler seg over tid. 
De fleste modellene basserer seg på ulike faser som en typisk gruppe går gjennom. 
Ikke alle grupper går gjennom alle fasene, og det kan være vanskelig å identifisere hvilke fase en gruppe er i.
Til tross for at disse modellen er godt utviklet og forstått er det ingen modell som nøyaktig vil beskrive en gruppeprosess, men de hjelper oss å forstå de ulike forandringene som en gruppe går gjennom. 
Rolfsen \& Lenin fokuserer på følgende modeller:
\vspace{\secspace}

\textbf{Halvtidsmodellen}, utviklet av Gersick (1989), tar utgangspunkt i studenter som arbeider med prosjektoppgaver som en del av sitt universitetsstudium.
Denne modellen består av to faser, der fase 1 er første halvdelen av prosjektet og fase 2 er andre halvdelen. 
Halvveis i prosjekter er her når studentene oppfatter de er halvveis, ikke nødvendig halvveis mtp. arbeidsmengde. 
Det som kjennetegner halvtidsmodellen er at effektiviteten, strukturen og målene er betydelig høyere i siste hlvdel av prosjektet enn i første. 
\vspace{\secspace}

\textbf{Åtte faser}, utviklet av Rosen (1987), er en modell som består av åtte forskjellige faser som gruppen går gjennom.
Gruppen utvikler seg i de fire første fasene. Her øker også ytelsen etterhvert som gruppen tar form.
Etter disse fasene kommer to faser med veldig intensivt arbeid og høy ytelse, som følges av en stagnasjonsfase. 
Siste fase er en avslutningsfase eller en fornyelsesfase.
Denne modellen antar ikke en gitt rekkefølge på fasene, men tar høyde for at gruppen kan hoppe frem og tilbake mellom de ulike fasene iløpet av gruppens levetid. 
Svakheten er at \textit{åtte faser} passer best for grupper som går over lengre tid, og da ikke kan benyttes i alle tilfeller. 
\vspace{\secspace}

\textbf{Forming, storming, norming, performing}, utviklet av Tuckman \& Jensen (1977), består av fire faser og er en modell som prøver å forklare hvorfor det tar tid før at en gruppe blir produktiv.
\textit{Forming} er den første fasen og handler om å innhente informasjon. 
I denne fasen skal gruppen bli kjent, møtes og forme mål/mandat.
Fase nummer to kalles \textit{storming} og er en følelsesmessig fase. 
Her skal gruppen bli enige om hvordan oppgaven skal løses. 
Etterhvert som de lærer hverandre å kjenne finner de også sin plass/rolle i gruppen. 
Gruppetilhørigheten blir sterkere i fasen \textit{norming}. 
Til nå har gruppen etablert normer og regler som gjelder innad i gruppa. 
Siste fase kalles \textit{performing} og er den effektive fasen der mesteparten av problemstillingen blir løst. 

\section{Kommunikasjonsteori}
Det er viktig med god kommunikasjon innad i gruppen for at samarbeidet skal fungere. 
Med kommunkasjon menes utveksling av meninger, oppfatninger, tanker og følelser som skjer mellom mennesker.
\vspace{\secspace}

Kommunkasjonsprosessen består alltid av minst to roller, en sender og en/flere mottakere. 
Senderen skal formidle noe til mottakeren og dette skal skje uten missforståelser. 
Dette forutsetter at senderen selv vet hva som skal formidles og han/hun må kunne sette det over i tale eller skriftelig kommunikasjon. 
Ofte er det store problemet i kommunikasjon selve forståelsen til mottakeren. 
For at dette skal gå best mulig må sender tilpasse budskapet til mottakerens kompetanse, og mottakeren må kunne stille spørsmål for å oppklare eventuelle uklarheter. 
Derfor er det viktig at sender og mottaker arbeider sammen om kommunikasjonsprosessen.
\vspace{\secspace}

At en gruppe har et positivt kommunikasjonsklima vil si at medlemmene føler seg emosjonelt komfortable. 
I slike grupper blir det ofta saklige og åpne diskusjoner der alle kommer med sine ideer og spørsmål. 
Negativt kommunikasjonsklima er preget av defensiv adferd. 
Ofte kan medlemmer få støtende respons når de kommuniserer, som f.eks sarkasme eller negativ evaluering, og dette resulterer i mer og mer defensiv adferd. 
Vi sier derfor at et negativt kommunikasjonsklima er selvforsterkende. 
\vspace{\secspace}

For å forbedre kommunikasjonsferdighetene er det viktig å: Tenke over hvordan man stiller spørsmål, lytte aktivt, gi konstruktiv tilbakemelding og vise evne til å takle følelser. 
Hvilke spørsmål som bidrar til positiv kommunikasjon kan være så mangt. 
Unngå enkle ja/nei spørsmål og heller still spørsmål som oppmuntrer til diskusjon, da det ofte er bedre at gruppen diskuterer ulemper og fordel fremfor å si hvem som er enige eller uenige. 
Aktiv lytting vil si å stille oppklarende spørsmål underveis og gjøre slik at den som snakker føles seg hørt og forstått. 
Konstruktive tilbakemeldinger kan være både positive og negative. 
Det er viktige at den som får tilbakemeldingen ikke føler at det blir gitt kritikk, men heller opplyser om forbedringspotensiale. 

\section{SITRA-modellen}
SITRA modellen er en modell som tar sikte på å gjøre skillet mellom situasjoner, observasjoner og aksjoner tydeligere.
Hovedelementene er:

\begin{enumerate}
  \item \textbf{Situasjon} - Den situasjonen som observeres.
  \item \textbf{Teori} - 
  \item \textbf{Refleksjon} -
  \item \textbf{Aksjon} - 
\end{enumerate}

I gruppens tilfelle ble en begrenset versjon av SITRA-modellen brukt. 
Derfor inneholder grupperefleksjonene hovedsaklig de tre punktene, situasjon, refleksjon og ,i de tilfellene det trengs, aksjon.
Gruppen har bevisst prøvd å forme gruppeloggen etter SITRA-modellen, slik at det blir enklere å se på gruppens fremgang i ettertid. 

