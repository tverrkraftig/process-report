\section{Gruppeutvikling}
Det har vært gjort utallige forsøk på å beskrive hvordan en gruppe utvikler seg over tid. 
De fleste modellene basserer seg på ulike faser som en typisk gruppe går gjennom. 
Ikke alle grupper går gjennom alle fasene, og det kan være vanskelig å identifisere hvilke fase en gruppe er i.
Til tross for at disse modellen er godt utviklet og forstått er det ingen modell som nøyaktig vil beskrive en gruppeprosess, men de hjelper oss å forstå de ulike forandringene som en gruppe går gjennom. 
Rolfsen \& Lenin fokuserer på følgende modeller:
\vspace{\secspace}

\textbf{Halvtidsmodellen}, utviklet av Gersick (1989), tar utgangspunkt i studenter som arbeider med prosjektoppgaver som en del av sitt universitetsstudium.
Denne modellen består av to faser, der fase 1 er første halvdelen av prosjektet og fase 2 er andre halvdelen. 
Halvveis i prosjekter er her når studentene oppfatter de er halvveis, ikke nødvendig halvveis mtp. arbeidsmengde. 
Det som kjennetegner halvtidsmodellen er at effektiviteten, strukturen og målene er betydelig høyere i siste hlvdel av prosjektet enn i første. 
\vspace{\secspace}

\textbf{Åtte faser}, utviklet av Rosen (1987), er en modell som består av åtte forskjellige faser som gruppen går gjennom.
Gruppen utvikler seg i de fire første fasene. Her øker også ytelsen etterhvert som gruppen tar form.
Etter disse fasene kommer to faser med veldig intensivt arbeid og høy ytelse, som følges av en stagnasjonsfase. 
Siste fase er en avslutningsfase eller en fornyelsesfase.
Denne modellen antar ikke en gitt rekkefølge på fasene, men tar høyde for at gruppen kan hoppe frem og tilbake mellom de ulike fasene iløpet av gruppens levetid. 
Svakheten er at \textit{åtte faser} passer best for grupper som går over lengre tid, og da ikke kan benyttes i alle tilfeller. 
\vspace{\secspace}

\textbf{Forming, storming, norming, performing}, utviklet av Tuckman \& Jensen (1977), består av fire faser og er en modell som prøver å forklare hvorfor det tar tid før at en gruppe blir produktiv.
\textit{Forming} er den første fasen og handler om å innhente informasjon. 
I denne fasen skal gruppen bli kjent, møtes og forme mål/mandat.
Fase nummer to kalles \textit{storming} og er en følelsesmessig fase. 
Her skal gruppen bli enige om hvordan oppgaven skal løses. 
Etterhvert som de lærer hverandre å kjenne finner de også sin plass/rolle i gruppen. 
Gruppetilhørigheten blir sterkere i fasen \textit{norming}. 
Til nå har gruppen etablert normer og regler som gjelder innad i gruppa. 
Siste fase kalles \textit{performing} og er den effektive fasen der mesteparten av problemstillingen blir løst. 