\section{Gruppeteori}
Det finnes mange forskjellige definisjoner på en arbeidsgruppe/team. 
Definisjonen til Katzenbach og Smith lyder som følger:

\begin{center}
\textit{Et team defineres som flere personer som arbeider sammen for å oppnå et felles mål. 
Medlemmene må være avhengige av hverandre på en eller annen måte.}
\newline 
(Katzenbach \& Smith, 1993)
\end{center}

Gruppe settes med andre ord ofte sammen for å oppnå et mål som er vanskelig/umulig å oppnå alene. 
Det er ikke slik at det alltid er fordelmessig å jobbe i gruppe, men komplekse flerfaglige/uoversiktelige oppgaver blir gjerne enklere å løse i grupper.
Gruppearbeid kan også gi løsninger preget av en større grav av nyskapning og kreativitet. 
Generelt er den største motivasjonen for å arbeide i en gruppe at større prosjekter har en veldig stor grad av tverrfaglighet. 
Dette er noe som kommer godt frem i arbeidslivet også. 
En gruppe bør ikke bestå av mer enn 10 personer, men 5-6 regnes av Katzenbach \& Smith som den optimale størrelsen.
\vspace{\secspace}

En \textbf{homogen gruppe} vil si at gruppemedlemmene har liten faglig spredning. 
En slik gruppe vil være mest effektiv når prosjektet har begrenset omfang og er lite flerfaglig. 
Fordelen med en slik gruppe er at gruppemedlemmene kommer ofte godt overens og har derfor gjerne mindre sosiale utfordringer. 
Det er også noen bakdeler med en homogen gruppe. 
Homogene grupper har en tendens til å unngå risikoer, og dermed går glipp av muligheter. 
De har også problemer med å tilpasse seg dynamiske situasjoner.
\vspace{\secspace}

Det motsatte av en homogen gruppe er en \textbf{heterogen gruppe}. 
Dette er en gruppe preget faglig spredning og er ofte bedre til å ta viktige beslutninger samt tenke nytt og kreativt enn homogene grupper. 
Til tross for dette er det ikke gitt at en faglig spredt gruppe resulterer i suksess. 
Ofte kan heterogene grupper ha større utfordringer når det kommer til intern kommunikasjon eller problemer med at medlemmer blir deffansive og derfor mindre produktive. 
