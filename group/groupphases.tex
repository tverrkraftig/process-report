\section{Faser og effektivitet}
Det har vært gjort utallige forsøk på å beskrive hvordan en gruppe utvikler seg over tid. 
De fleste modellene basserer seg på ulike faser som en typisk gruppe går gjennom. 
Ikke alle grupper går gjennom alle fasene, og det kan være vanskelig å identifisere hvilke fase en gruppe er i.
Til tross for at disse modellen er godt utviklet og forstått er det ingen modell som nøyaktig vil beskrive en gruppeprosess, men de hjelper oss å forstå de ulike forandringene som en gruppe går gjennom. 
Ofte kan en gruppes utvikling beskrives av en kombinasjon av diverse modeller. 
\vspace{\secspace}

BØR MODELLENE BESKRIVES HER???

%\begin{figure}
%    \begin{tikzpicture}[remember picture, overlay, xshift=-0.5cm, yshift=14.3cm]
%        \foreach \x in {0,-1.5,-3,-4.5,-6,-7.5,-9,-10.5,-12,-13.5}%,-15,-16.5}
%            \draw [color=white, fill=LightGray, blur shadow] (0,\x) -- (0.5,\x+0.5) -- (0.5,\x-1) -- (0,\x-1.5) -- (-0.5,\x-1) -- (-0.5,\x+0.5) -- (0,\x);
%        \draw [color=white, top color=red, bottom color=DarkRed, blur shadow] (0,0) -- (1,1) -- (1,-2) -- (0,-3) -- (-1,-2) -- (-1,1) -- (0,0);
%        \draw (0,-0.5) node[anchor=north, color=white] {\Large \textbf{Fase 1}};
%        \draw [color=white, top color=red, bottom color=DarkRed, blur shadow] (0,0-9) -- (1,1-9) -- (1,-2-9) -- (0,-3-9) -- (-1,-2-9) -- (-1,1-9) -- (0,0-9);
%        \draw (0,-0.5-9) node[anchor=north, color=white] {\Large \textbf{Fase 2}};
        %\draw (0,0) -- (0,-20);
%    \end{tikzpicture}
%\end{figure}

%\begin{adjustwidth}{1cm}{0cm}
% Text here
%\end{adjustwidth}
