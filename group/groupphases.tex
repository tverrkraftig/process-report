\section{Faser og effektivitet}
Det har vært gjort utallige forsøk på å beskrive hvordan en gruppe utvikler seg over tid. 
De fleste modellene basserer seg på ulike faser som en typisk gruppe går gjennom. 
Ikke alle grupper går gjennom alle fasene, og det kan være vanskelig å identifisere hvilke fase en gruppe er i.
Til tross for at disse modellen er godt utviklet og forstått er det ingen modell som nøyaktig vil beskrive en gruppeprosess, men de hjelper oss å forstå de ulike forandringene som en gruppe går gjennom. 
Ofte kan en gruppes utvikling beskrives av en kombinasjon av diverse modeller. 
\vspace{\secspace}

\textit{Halvtidsmodellen}, utviklet av Gersick (1989), tar utgangspunkt i studenter som arbeider med prosjektoppgaver som en del av sitt universitetsstudium.
Ofte blir denne modellen nevnt i forbindelse med prosjektoppgaver i studiet. 
Denne modellen består av to faser, der fase 1 er første halvdelen av prosjektet og fase 2 er andre halvdelen. 
Halvveis i prosjekter er her når studentene oppfatter de er halvveis, ikke nødvendig halvveis med tanke på arbeidsmendge. 
Det som kjennetegner halvtidsmodellen er at effektiviteten, strukturen og målene er betydelig høyere i siste halvdel av prosjektet enn i første.
Gruppen mener denne modellen ble for snever for dette tilfellet, men siden den er mye nevnt i lignende sammenhenger var det logisk å beskrive den her. 
\vspace{\secspace}

En modell som beskriver gruppens utvikling godt er \textit{Forming, norming, storming, performing}, som er en modell utviklet av Tickman \& Jensen (1977). 
Denne modellene søker å beskrive hvorfor det tar tid før en gruppe blir produktiv. 
\textit{Forming} er den første fasen og handler om å innhente informasjon. 
I denne fasen skal gruppen bli kjent, møtes og forme mål/mandat.
Fase nummer to kalles \textit{storming} og er en følelsesmessig fase. 
Her skal gruppen bli enige om hvordan oppgaven skal løses. 
Etterhvert som de lærer hverandre å kjenne finner de også sin plass/rolle i gruppen. 
Gruppetilhørigheten blir sterkere i fasen \textit{norming}. 
Til nå har gruppen etablert normer og regler som gjelder innad i gruppa. 
Siste fase kalles \textit{performing} og er den effektive fasen der mesteparten av problemstillingen blir løst. 
\vspace{\secspace}

\subsection{Fasene}

%\begin{figure}
%    \begin{tikzpicture}[remember picture, overlay, xshift=-0.5cm, yshift=14.3cm]
%        \foreach \x in {0,-1.5,-3,-4.5,-6,-7.5,-9,-10.5,-12,-13.5}%,-15,-16.5}
%            \draw [color=white, fill=LightGray, blur shadow] (0,\x) -- (0.5,\x+0.5) -- (0.5,\x-1) -- (0,\x-1.5) -- (-0.5,\x-1) -- (-0.5,\x+0.5) -- (0,\x);
%        \draw [color=white, top color=red, bottom color=DarkRed, blur shadow] (0,0) -- (1,1) -- (1,-2) -- (0,-3) -- (-1,-2) -- (-1,1) -- (0,0);
%        \draw (0,-0.5) node[anchor=north, color=white] {\Large \textbf{Fase 1}};
%        \draw [color=white, top color=red, bottom color=DarkRed, blur shadow] (0,0-9) -- (1,1-9) -- (1,-2-9) -- (0,-3-9) -- (-1,-2-9) -- (-1,1-9) -- (0,0-9);
%        \draw (0,-0.5-9) node[anchor=north, color=white] {\Large \textbf{Fase 2}};
        %\draw (0,0) -- (0,-20);
%    \end{tikzpicture}
%\end{figure}

%\begin{adjustwidth}{1cm}{0cm}
% Text here
%\end{adjustwidth}
