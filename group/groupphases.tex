\section{Faser og effektivitet}
Det har vært gjort utallige forsøk på å beskrive hvordan en gruppe utvikler seg over tid. 
De fleste modellene basserer seg på ulike faser som en typisk gruppe går gjennom. 
Ikke alle grupper går gjennom alle fasene, og det kan være vanskelig å identifisere hvilke fase en gruppe er i.
Til tross for at disse modellen er godt utviklet og forstått er det ingen modell som nøyaktig vil beskrive en gruppeprosess, men de hjelper oss å forstå de ulike forandringene som en gruppe går gjennom. 
Ofte kan en gruppes utvikling beskrives av en kombinasjon av diverse modeller (Levin 2012). 
\vspace{\secspace}

\textit{Halvtidsmodellen}, utviklet av Gersick (1989), tar utgangspunkt i studenter som arbeider med prosjektoppgaver som en del av sitt universitetsstudium.
Ofte blir denne modellen nevnt i forbindelse med prosjektoppgaver i studiet. 
Denne modellen består av to faser, der fase 1 er første halvdelen av prosjektet og fase 2 er andre halvdelen. 
Halvveis i prosjekter er her når studentene oppfatter de er halvveis, ikke nødvendig halvveis med tanke på arbeidsmendge. 
Det som kjennetegner halvtidsmodellen er at effektiviteten, strukturen og målene er betydelig høyere i siste halvdel av prosjektet enn i første.
Gruppen mener denne modellen ble for snever for dette tilfellet, men siden den er mye nevnt i lignende sammenhenger var det logisk å beskrive den her. 
\vspace{\secspace}

En modell som beskriver gruppens utvikling godt er \textit{Forming, norming, storming, performing}, som er en modell utviklet av Tuckman \& Jensen (1977). 
Denne modellene søker å beskrive hvorfor det tar tid før en gruppe blir produktiv. 
\textit{Forming} er den første fasen og handler om å innhente informasjon. 
I denne fasen skal gruppen bli kjent, møtes og forme mål/mandat.
Fase nummer to kalles \textit{storming} og er en følelsesmessig fase. 
Her skal gruppen bli enige om hvordan oppgaven skal løses. 
Etterhvert som de lærer hverandre å kjenne finner de også sin plass/rolle i gruppen. 
Gruppetilhørigheten blir sterkere i fasen \textit{norming}. 
Til nå har gruppen etablert normer og regler som gjelder innad i gruppa. 
Siste fase kalles \textit{performing} og er den effektive fasen der mesteparten av problemstillingen blir løst. 
\vspace{\secspace}

\subsection{Fasene}
Som nevnt over kan ikke hele gruppearbeidet beskrives av de modellene som finnes. 
Utviklingen vil være en kombinasjon av de forskjellige modellene og når fasene starter/stopper vil ofte være ''flytende''.
Det vil si at det ikke nødvendigvis er en ''slutt-til-start''-sammenheng mellom de, men gruppen kan befinne seg i flere faser samtidig. 
Under vil gruppens utvikling bli forklart og forsøkt knyttet opp mot de dominerende modellene som beskriver en typisk utvikling. 
Vi har derfor valgt å kun fokusere på de fasene som skiller seg tydelig fra hverandre.
\vspace{\secspace}

%% Fasene vi gikk gjennom 

%%%%%%%%%%%%%%%%%%%%%%%%%%%%%%%%%%%%%%%%%%%%%%%%%%%%%%%
%% Piler side 1
\begin{figure}
    \begin{tikzpicture}[remember picture, overlay, xshift=1cm, yshift=-15.3cm]
        \foreach \x in {0,-1.5,-3,-4.5}%,-6,-7.5,-9,-10.5,-12,-13.5,-15,-16.5}
            \draw [color=white, fill=LightGray, blur shadow] (0,\x) -- (0.5,\x+0.5) -- (0.5,\x-1) -- (0,\x-1.5) -- (-0.5,\x-1) -- (-0.5,\x+0.5) -- (0,\x);
        \draw [color=white, top color=red, bottom color=DarkRed, blur shadow] (0,0) -- (1,1) -- (1,-2) -- (0,-3) -- (-1,-2) -- (-1,1) -- (0,0);
        \draw (0,-0.5) node[anchor=north, color=white] {\Large \textbf{Fase 1}}; 
    \end{tikzpicture}
\end{figure}
\begin{adjustwidth}{2.5cm}{0cm}

\textbf{\Large Fase 1 - Bli kjent}
I begynnelsen av gruppearbeidet skulle gruppen bli kjent med hverandre, dette markerer begynnelse på første fase.
De fleste modeller har en form for initieringsfase der gruppemedlemmene møtes for første gang og stifter bekjentskap. 
Fasilitatorene kjørte noen \textit{bli-kjent}-øvelser som hjalp til med å raskere lære hverandre å kjenne. 
Denne fasen begynte 1. landsbydag og var preget av lite faglig aktivitet. 
Gruppen diskuterte også hvilke forventninger de hadde til prosessbiten og samarbeidet. 
Det ble også kartlagt hvilke kunnskaper og tanker gruppen hadde om den tekniske delen av prosjektet. 
Det var dette som skulle danne grunnlaget for neste fase. 

Johnson \& Johnson (2013) legger vekt på at for å utvikle et effektivt gruppesamarbeid er det viktig at følgende betingelser er oppfyllt:
\begin{enumerate}
    \item Klare og veldefinerte mål
    \item God toveis kommunikasjon
    \item Lederskap og deltakelse noenlunde likt mellom medlemmene
\end{enumerate} 
Gruppen var flink med god kommunikasjon tidlig i prosjektet, men mål og mer lik deltakelse kom med tiden. 
En forklaring på dette kan være at gruppen ikke fikk snakket så mye med hverandre første dagen, da den gikk bort til mange felles ''bli-kjent''-øvelser. 
\vspace{\secspace}

%%%%%%%%%%%%%%%%%%%%%%%%%%%%%%%%%%%%%%%%%%%%%%%%%%%%%%%
%% Piler side 2
\begin{figure}
    \begin{tikzpicture}[remember picture, overlay, xshift=1cm, yshift=1cm]
        \foreach \x in {0,-1.5,-3,-4.5,-6,-7.5,-9,-10.5,-12,-13.5,-15,-16.5,-18,-19.5}
            \draw [color=white, fill=LightGray, blur shadow] (0,\x) -- (0.5,\x+0.5) -- (0.5,\x-1) -- (0,\x-1.5) -- (-0.5,\x-1) -- (-0.5,\x+0.5) -- (0,\x);
        \draw [color=white, top color=red, bottom color=DarkRed, blur shadow] (0,0-7.5) -- (1,1-7.5) -- (1,-2-7.5) -- (0,-3-7.5) -- (-1,-2-7.5) -- (-1,1-7.5) -- (0,0-7.5);
        \draw (0,-0.5-7.5) node[anchor=north, color=white] {\Large \textbf{Fase 2}};
        \draw [color=white, top color=red, bottom color=DarkRed, blur shadow] (0,0-16.5) -- (1,1-16.5) -- (1,-2-16.5) -- (0,-3-16.5) -- (-1,-2-16.5) -- (-1,1-16.5) -- (0,0-16.5);
        \draw (0,-0.5-16.5) node[anchor=north, color=white] {\Large \textbf{Fase 3}};
    \end{tikzpicture}
\end{figure}

\noindent \textbf{\Large Fase 2 - Brainstorming}
Fase 2 var en fase som begynte før gruppen var kommet ut av fase 1. 
2. landsbydag ble det presentert noen alternative oppgaver som vi kunne velge, dette resulterte i at gruppen var usikker på hvilken oppgave vi ville ha. 
For å lettere kunne ta en avgjørelse ble det bestemt at begge de to alternativene skulle diksuteres og drøftes, slik at det ble enklere å ta en beslutning. 
Denne brainstormingen bestod av åpne diskusjoner der alle kunne si sin mening om alternativene. 
Det var meningen at disse diskusjonene skulle være så åpne som mulig, slik at alle fikk komme med sine meninger og ideer. 
Gruppen bestemte til slutt seg for en av oppgavene basert på det som ble diskutert. 
Samtidig som dette ble gjort, lærte medlemmene hverandre å kjenne. 
De interne reglene som er beskrevet i samarbeidsavtalen ble formet og vi begynte å følge disse. 

De to første fasene ligner veldig på begynnelsen av modellen til Tuckman \& Jensen, \textit{Forming, norming, storming, performing}. 
Denne modellen legger vekt på at prosjektets mål/mandat og gruppens normer/regler formes i begynnelsen av prosjektet. 
Normer kan beskriv som hvordan de andre \textit{forventer} at medlemmene skal oppføre seg (Schwarz 2002). 
At alle var med på å forme problemstillingen etter det de kunne/ville lære mener vi hjalp til med å holde motivasjonen oppe lengre ut i prosjektet. 
\vspace{\secspace}

\noindent \textbf{\Large Fase 3 - ''Somlefasen''}
Etter at problemstillingen var formet var prosjektet preget av en fase der ting tok tid. 
Det var deler av prosjektet som var avhengige av at noen komponeneter fungerte sammen med gitt hardware, og det ble derfor en del knoting før dette fungerte.  
I gruppeloggen fremkommer det at flere var litt misfornøyde med framgangen i denne fasen. 
Dette var heller ikke en fase alle gruppemedlemmene var med på. 
Oppdelingen av prosjektet gjorde at de som ikke jobbet med denne delen kunne jobbe uavhengig av somlingen. 
Slik oppretholdte gruppen en bedre fremgang enn den ellers ville klart. 

Dette er ikke en fase man vanligvis ser så tidlig i prosjektet, men kan sammenlinges med stagnasjonsfasen i \textit{Åtte faser}-modellen av Rosen (1987).
Ofte kan en slike fase virke demotiverende for hele gruppen. 
Derfor er det viktig at gruppen raskest mulig ser problemet og prøver å løse det. 
I dette tilfellet ble det ikke fattet noen direkte aksjoner som følger av stagnasjonen, noe som kanskje burde vært gjort for å forbedre effektiviteten. 
\vspace{\secspace}

%%%%%%%%%%%%%%%%%%%%%%%%%%%%%%%%%%%%%%%%%%%%%%%%%%%%%%%
% Piler side 3
\begin{figure}
    \begin{tikzpicture}[remember picture, overlay, xshift=1cm, yshift=1cm]
        \foreach \x in {0,-1.5,-3,-4.5,-6,-7.5,-9,-10.5,-12,-13.5,-15}%,-16.5,-18,-19.5}
            \draw [color=white, fill=LightGray, blur shadow] (0,\x) -- (0.5,\x+0.5) -- (0.5,\x-1) -- (0,\x-1.5) -- (-0.5,\x-1) -- (-0.5,\x+0.5) -- (0,\x);
        \draw [color=white, top color=red, bottom color=DarkRed, blur shadow] (0,0-6) -- (1,1-6) -- (1,-2-6) -- (0,-3-6) -- (-1,-2-6) -- (-1,1-6) -- (0,0-6);
        \draw (0,-0.5-6) node[anchor=north, color=white] {\Large \textbf{Fase 4}};
        \draw [color=white, top color=red, bottom color=DarkRed, blur shadow] (0,0-13.5) -- (1,1-13.5) -- (1,-2-13.5) -- (0,-3-13.5) -- (-1,-2-13.5) -- (-1,1-13.5) -- (0,0-13.5);
        \draw (0,-0.5-13.5) node[anchor=north, color=white] {\Large \textbf{Fase 5}}; 
    \end{tikzpicture}
\end{figure}


\noindent \textbf{\Large Fase 4 - Performing}
På grunn av at den forrige fasen ikke påvirket hele gruppen i like stor grad, kom Fase 4 på litt forskjellig tidspunkt for de ulike gruppemedlemmene. 
Kan i grove trekk si at gruppen var i performing-fasen etter 5. landsbydag. 
Da var det tekniske som bød på problemer løsnet, og hele gruppen var godt i gang med arbiedsoppgavene sine. 
Her sto produktet i hovedfokus, noe som resulterte i færre diskusjoner innad i gruppa og mindre tid til prosess. 
Til tross for at det ble mindre tid til prosess var gruppen flink til å revidere de aksjoner som ble vedtatt. 
Spesielt pause-tiden hadde en tendens til å være litt flytende. 
Dette ble forsøkt rettet med fast felles lunch, men i ettertid ser vi at det ikke funket så bra som vi håpet på. 

Performing-fasen var en effektiv fase der mye av arbeidet ble gjort og er en fase som går igjen i de fleste modeller. 
I og med at denne fasen kom på midten av prosjektet kan den også knyttes til halvtidsmodellen, men gruppen mener det skyldes mer tilfeldigheter at den intraff akkurat på dette tidspunktet. 
Også \textit{Forming, norming, storming, performing} og \textit{Åtte faser} inneholder en slik effektiv fase. 
\vspace{\secspace}

\noindent \textbf{\Large Fase 5 - Avslutning}
Til tross for at abriedet stoppet litt opp i påskeferien velger gruppen å ikke se på dette som en stagnasjonsfase. Ferie er ferie. 
Det vil si at etter performing kom avslutningsfasen. 
I denne fasen ble prosjektet finpusset litt på og vist fram til resten av landsbyen. 
Gruppen hadde prøvd å ta gode notater iløpet av prosjektet, men nå begynte arbeidet med å skrive det inn i rapportene. 
Derfor ble avslutningsfasen preget av mye skriving, både teknisk og prosess. 
\vspace{\secspace}
\end{adjustwidth}

\vspace{\secspace}
Ut fra de fasene som gruppen har vært gjennom kommer det fram at gruppearbeidet ikke nødvendigvis har fulgt en modell gjennom hele prosjektet. 
På grunn av at gruppearbeidet er så kompleks og alle tilfellene er forskjellige er det umulig å bruke kun en modell (Levin 2012).
Vårt tilfelle beskrives best som en blanding av \textit{Forming, norming, storming, performing}, \textit{åtte faser} og \textit{halvtidsmodellen}. 
Dette på grunn av at de første to fasene lignet på begynnelsen av FNSP og \textit{åtte faser}, samt det tydelige vendepunktet som kom ca halvveis i prosjektet. 
Også den økende ytelsen er typisk i Rosens \textit{åtte faser}. 
Det fremkommer klart at ingen av de ulike modellene alene kan brukes til å beskrive gruppens utvikling gjennom prosjektet. 
