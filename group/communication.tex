\section{Kommunikasjon}
Det er viktig med god kommunikasjon innad i gruppen for at samarbeidet skal fungere. 
Med kommunkasjon menes utveksling av meninger, oppfatninger, tanker og følelser som skjer mellom mennesker.
Å kommunisere er å gjøre et forsøk på skape en lignende sinnstilstand som man selv har, hos noen andre (Levin 2012).
\vspace{\secspace}

Kommunkasjonsprosessen består alltid av minst to roller, en sender og en/flere mottakere. 
Senderen skal formidle noe til mottakeren og dette skal skje uten missforståelser. 
Dette forutsetter at senderen selv vet hva som skal formidles og han/hun må kunne sette det over i tale eller skriftelig kommunikasjon. 
Ofte er det store problemet i kommunikasjon selve forståelsen til mottakeren. 
For at dette skal gå best mulig må sender tilpasse budskapet til mottakerens kompetanse, og mottakeren må kunne stille spørsmål for å oppklare eventuelle uklarheter. 
Derfor er det viktig at sender og mottaker arbeider sammen om kommunikasjonsprosessen (Levin 2012).

\subsection{Utgangpunktet}
Hvordan var utgangspunktet?
Eksempler på situasjoner som illustrerer dette. 
Noe teori? Kan finne teori i mappen theory/communication. 

\subsection{Underveis}
Hva som skjedde underveis? 
Hvordan utviklet det seg? 
Eksempler som illustrerer UTVIKLLINGEN. 

\subsection{Siste landsbydagene}
De to siste landsbydagene.
Har vi utvilklet oss positivt eller negativt? Begge deler? 
Eksempler som knyttes mot teori her ogspå! 

