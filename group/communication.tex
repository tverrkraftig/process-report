\section{Kommunikasjon}
Det er viktig med god kommunikasjon innad i gruppen for at samarbeidet skal fungere. 
Med kommunikasjon menes utveksling av meninger, oppfatninger, tanker og følelser som skjer mellom mennesker.
Å kommunisere er å gjøre et forsøk på skape en lignende sinnstilstand som man selv har, hos noen andre\citep{levin}.
\vspace{\secspace}

Kommunikasjonsprosessen består alltid av minst to roller, en sender og en/flere mottakere. 
Senderen skal formidle noe til mottakeren og dette skal skje uten misforståelser. 
Dette forutsetter at senderen selv vet hva som skal formidles og han/hun må kunne sette det over i tale eller skriftlig kommunikasjon. 
Ofte er det store problemet i kommunikasjon selve forståelsen til mottakeren. 
For at det skal gå best mulig må sender tilpasse budskapet til mottakerens kompetanse, og mottakeren må kunne stille spørsmål for å oppklare eventuelle uklarheter. 
Derfor er det viktig at sender og mottaker arbeider sammen om kommunikasjonsprosessen \citep{levin}.

\subsection{Utgangpunktet}
Som beskrevet i forrige kapittel var det første møtet preget av en ''bli kjent''-fase. 
Noen av gruppemedlemmene kjente hverandre fra før, mens andre møtte hverandre for første gang. 
De første landsbydagene var preget av mange diskusjoner og tanker om hva den tekniske oppgaven skulle handle om. 
Dette resulterte naturlig nok i mye kommunikasjon og kommunikasjonsmønstret i gruppen kom godt fram. 
\vspace{\secspace}

Til å begynne med var det enkelte i gruppen som var mer framtredende enn andre. 
Anders og Odd var de to i gruppen som kom mest fram. 
Anders trodde dette kom av at han var vant til å ta lederrollen og har deltatt aktivt i lederverv før (samfundet, etc).
Dette gjorde at mange diskusjoner ble preget av at Anders og Odd tok mye styring. 
Det var for eksempel noen ganger samtalen gikk mer mellom disse to, enn mellom hele gruppen som helhet. 
Denne observasjonen stemmer veldig bra med Meyers-Briggs personlighetstest, som ble presentert i kapittel~\ref{chap:meyersgroupmembers}.
Meyers-Briggs-testen slo fast at Anders og Odd hadde preferanser for utadvent(Extraversion), mens resten av gruppen hadde preferanser for innadvendt(Introvert). 
\vspace{\secspace}

Til tross for at samtalene ble dominert av Anders og Odd, var hele gruppen flinke til å høre på de andre. 
Det var ingen som følte at de ble overkjørt av de andre når de ville si noe. 
Gruppen valgte å la alle komme med sin mening før noe ble bestemt. 
Dette ble praktisert med en runde rundt bordet der alle fikk si hva de tenkte. 
En gruppe med positivt kommunikasjonsklima har åpne diskusjoner der alle i gruppen får komme med sine tanker og ideer\citep{levin}. 
Et negativt kommunikasjonsklima preges derimot av defansiv adferd og støtende respons. 
2. landsbydag, da gruppen ble fasilitert, tok Anders opp problematikken med at han og Odd dominerte i gruppen. 
De andre var enige i at dette var en tendens i gruppen, og Petter mente det kunne skyldes at det enda var tidlig i gruppearbeidet. 
Det var bra at dette ble tatt opp siden konstruktive tilbakemeldinger er svært viktig for å forbedre kommunikasjonsklima innad i gruppen.
\citet{levin} understreker også at tilbakemeldingene skal også gis på en slik måte at ingen føler seg angrepet, men heller føler at det opplyses om forbedringspotensiale.
Gruppen ble derfor enige om at det var hensiktmessig med en leder-/ordstyrerrolle som rullerte fra gang til gang.
Dette ble ført inn som et eget punkt i samarbeidsavtalen(se Vedlegg 1). 
Slik ville alle i gruppen få bedre læringsutbytte iløpet av samarbeidet i EiT. 
\vspace{\secspace}

\subsection{Underveis}
Landsbyens tema, Instrumentering og Styring Over Internett, er veldig teknisk. 
Ettersom arbeidsoppgavene ble fordelt ut fra hva vi kunne og hva vi ville lære, ble det mindre diskusjoner innad i gruppen etter at den tekniske oppgaven var bestemt. 
Diskusjonene oppsto derfor i all hovedsak under fasilitering, innsjekk, grupperefleksjon og pauser. 
\vspace{\secspace}

Til tross for gruppens homogenitet er det av og til fagterminologi som ikke hele gruppen er inneforstått med.  
Tidlig i prosjektet skulle Emil forklare noe om server delen. 
Emil la merke til at de andre ikke forsto hva han mente, og derfor forklarte han det på nytt. 
Gruppen var enige om at dette var et godt eksempel på hvor ting fungerte bra. 
Det er viktig at både den som forklarer noe og den som lytter aktivt setter seg inn i den andres situasjon.
\citet{levin} forklarer at for å bedre kommunikasjonen må den som lytter stille oppklarende spørsmål, mens den som forteller må tenke på hvem som er mottakeren. 
Det er like viktig at den som forklarer noe føler seg hørt og forstått, som at mottaker fortår hva som blir forklart. 
Siden dette fungerte bra ble det ikke gjort noen aksjoner her, men det ble oppfordret til å opprettholde den lave terskelen for å forklare noe på nytt eller stille oppklarende spørsmål.
\vspace{\secspace}

Et problem som gruppen straks støtte på var at diskusjoner kunne gli ut i usakeligheter. 
Det var en rekke situasjoner der vi diskuterte oss langt ut på viddene. 
Selv om det kanskje ikke virker som et stort problem der og da er det viktig å huske at et godt kommunikasjonsklima preges av saklige diskusjoner\citep{levin}. 
Emil mente at disse usakelige diskusjonene tok mye tid, og derfor begrenset effektiviteten. 
Dette var resten av gruppen enige i. 
Inspirert av videoforelesningene i faget\citep{video} ble det bestemt at et \textit{time out}-tegn skulle benyttes i slike situasjoner. 
Etter at noen sa ''time out'' skulle den usakelige diskusjonene opphøre med en gang, og gruppen gå tilbake til det opprinnelige samtaleemnet. 
\vspace{\secspace}

Når gruppen var kommet i gang med det tekniske arbeidet, ble det på et tidspunkt prøvd på å sitte på to forskjellige bord. 
Dette fungerte i praksiss bra for den tekniske delen, men var kanskje litt dumt når det gjelder prosessdelen. 
Emil mente dette økte terskelen for å stille spørsmål, noe som kan sette en demper for både produktivitet og samhold. 
Gruppen var enige med dette og bestemte seg for å heller dette to bord sammen, slik at alle kunne sitte samlet. 
Dette gav gruppen bedre resultater og gruppesamholdet ble bedre.
\vspace{\secspace}

Det tekniske prosjekte dreide seg om å styre en servomotor fra en datamaskin. 
Dette kan gjøres på to forskjellige måter, via en USB-enhet eller vi UART-utgangen på datamaskinen. 
Ole og Petter arbeidet med dette under antagelsen at UART-utgangen på pc'en skulle benyttes. 
Etter flere timer med problemer ble dette nevnt i gruppen. 
De andre i gruppen reagerte på at istedenfor å holde på med UART, ville det være mye enklere å bruke USB-enheten. 
Det hadde med andre ord oppstått en misfortåelse som kostet gruppen noen timer med arbeid. 
Ole mente dette skjedde fordi at det ble nevnt at pc'en hadde UART og da ble det tatt for gitt at det var UART som skulle brukes. 
Emil mente at det at vi satt adskilt kanskje kunne ha litt av skylden for denne hendelsen. 
Når gruppen sitter sammen blir terskelen for å stille spørsmål lavere og denne hendelsen kunne vært unngått. 
Gruppen ble enige om at vi skulle prøve å sitte sammen fra nå av, og lederen skulle ha ansvar om at ikke diskusjonene ble avsluttet uten en konklusjon. 
\vspace{\secspace}

Det ble nevnt over at en rullerende lederrolle var en aksjon for at alle skulle bli mer aktive. 
Denne aksjonen fungerte generelt sett bra. 
Odd kommenterte at Petter ble mer aktiv når han hadde lederrollen. 
Dette gjaldt også de andre medlemmene som var mindre aktive i utgangspunktet. 
Denne ordningen ble opprettholdt gjennom resten av EiT. 
\vspace{\secspace}

\subsection{Resultatet}
For å kunne vise hvordan kommunikasjonsklimaet utviklet seg er må man se på gruppens situasjon mot slutten av prosjektet. 
På grunn av at EiT består av kun 15 landsbydager har vi valgt å se på de to siste landsbydagene for å evaluere kommunikasjonen i sluttfasen. 
Dette var nok en fordel i og med at det var disse to dagene alles arbeid skulle settes sammen, og dermed var disse dagene preget av mer kommunikasjon enn ellers.
\vspace{\secspace}

Gruppens tendenser til å skli ut har hatt periodiske tendenser gjennom hele prosjektet. 
Selv de siste landsbydagene har gruppen av og til blitt usakelige. 
Den store forskjellen er nok at gruppen har blitt mindre flink til å bruke ''time-out''-tegnet. 
Dette kan nok skyldes at gruppen alt i alt er litt lei av EiT og har mye annet å styre med. 
De aksjoner som ble gjort i forbindelse med dette problemet har ikke vært tilstrekkelig til å få bort problemet. 
Det kan skyldes at hvem som har ansvaret for ''time-out''-tegnet var ikke definert.
Slik følte ingen at de måtte ta ansvar for å vise tegnet. 
\vspace{\secspace}

Generelt sett har gruppens kommunikasjonsmønster utviklet seg positivt. 
Alle saklige diskusjoner har blitt praktisert på en slik måte at alle som vill si noe ble hørt. 
Dette ble gjort gjennom hele gruppen diskuterte fordeler og ulemper fremfor å si hvem som er enige eller uenige. 
Ved å praktisere dette bidrar man til et bedre kommunikasjonsklima\citep{levin}.
Andre ting som forbedret seg var terskelen for å ta opp ting i gruppen. 
Dette gjelder tekniske spørsmål så vel som problemer gruppemedlemmene imellom. 
Også aktivitetsnivået gjevnet seg ut. 
I begynnelsen var Anders og Odd veldig framtredende, mens Emil, Ole og Petter var litt mer stille. 
Nå, etter tiltakene og bevisstgjøringen på dette, er deltakelsen i kommunikasjonen mye jevnere.  
\vspace{\secspace}
