\section{SITRA-modellen}
SITRA modellen er en modell som tar sikte på å gjøre skillet mellom situasjoner, observasjoner og aksjoner tydeligere.
Hovedelementene er:

\begin{enumerate}
  \item \textbf{Situasjon} - Den situasjonen som observeres.
  \item \textbf{Teori} - Relevant teori som kan knyttes til sitasjonen.
  \item \textbf{Refleksjon} - Gruppens tanker rundt situasjonen.
  \item \textbf{Aksjon} - Noe som gruppen gjør for å forbedre dette. 
\end{enumerate}

I gruppens tilfelle ble en begrenset versjon av SITRA-modellen brukt. 
Derfor inneholder grupperefleksjonene hovedsaklig de tre punktene, situasjon, refleksjon og ,i de tilfellene det trengs, aksjon.
Gruppen har bevisst prøvd å forme gruppeloggen etter SITRA-modellen, slik at det blir enklere å se på gruppens fremgang i ettertid. 
Slik blir det også enklere å knytte inn relevant teori når gruppens utvikling analyseres. 