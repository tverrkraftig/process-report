\chapter{Konklusjon}

Etter et helt semester med Eksperter i Team, sitter gruppen igjen med mange erfaringer. 
Vi har alle fått prøve oss i lederrollen, og nøye reflektert over og diskutert hva denne rollen bør inneholde. 
I vårt tilfelle var dette å ha oversikt over dagens arbeid, og å skape en god struktur for gjennomføringen. 
Lederen må være pådriveren for effektive jobbing, og skjære gjennom når diskusjoner ikke lenger er produktive. 
Det å rullere lederrollen har vært en suksess når det gjelder læringsutbytte, men det kom tydelig frem at det gikk ut over effektiviten til gruppa. 
Dette var noe vi i utgangspunktet forventet, og valgte å prioritere økt innsikt i gruppedynamikk.

Vi har diskutert mye rundt hvordan arbeidet utføres, særlig når alle jobbber mer eller mindre selvstending. 
Vår konklusjon på dette området er at adskilte arbeidsposisjoner minsker samarbeidet og gruppesamholdet.
Det hever også listen for å spørre om små ting, noe som fører til at man heller bruker lang tid på å finne det ut selv. 
Det viktigste momentet var her at man ikke sikkert på to forskjellige bord med ryggen til hverandre, men at man heller danner et større bord som alle sitter rundt.


