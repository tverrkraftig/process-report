\begin{table}[H]
\begin{tabular}{l l}
        Navn: & Anders Strand \\
        Alder: & 24 år \\ 
        Linje: & Teknisk Kybernetikk \\
        Styrke: & Tar initiativ \\
        Svakhet: & Tidvis dominerende
    \end{tabular}
\end{table}

Mine første tanker om Eit kom hovedsakelig fra venner og kjente som har hatt faget før.
De hadde hatt både gode og dårlige opplevelser, og jeg fikk tips til hvordan jeg kunne få mest mulig ut av faget. 
Jeg brukte denne informasjonen når jeg søkte landsby, og fikk mulighet til å prege prosjektet i stor grad. 
Jeg fikk derfor ha en prosjekt jeg synes var spennende og lærerikt.
Prosessdelen av faget var jeg spent på, men hadde ikke den negative innstillingen flere hadde til å begynne med. 
Jeg var åpen for å lære mye om samarbeid, og lære meg selv bedre å kjenne. 
Jeg kan relatere disse emnene opp mot frivillige verv jeg har på fritiden, og også faget Ledelse i Praksis som jeg tar samme semester.
