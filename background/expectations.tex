\section{Gruppens forventninger til EiT}

%% Anders %%
\begin{table}[H]
\begin{tabular}{l l}
        Navn: & Anders Strand \\
        Alder: & 24 år \\ 
        Linje: & Teknisk Kybernetikk \\
        Styrke: & Tar initiativ \\
        Svakhet: & Tidvis dominerende
    \end{tabular}
\end{table}

Mine første tanker om Eit kom hovedsakelig fra venner og kjente som har hatt faget før.
De hadde hatt både gode og dårlige opplevelser, og jeg fikk tips til hvordan jeg kunne få mest mulig ut av faget. 
Jeg brukte denne informasjonen når jeg søkte landsby, og fikk mulighet til å prege prosjektet i stor grad. 
Jeg fikk derfor ha en prosjekt jeg synes var spennende og lærerikt.
Prosessdelen av faget var jeg spent på, men hadde ikke den negative innstillingen flere hadde til å begynne med. 
Jeg var åpen for å lære mye om samarbeid, og lære meg selv bedre å kjenne. 
Jeg kan relatere disse emnene opp mot frivillige verv jeg har på fritiden, og også faget Ledelse i Praksis som jeg tar samme semester.

%% Petter %%
\begin{table}[H]
    \begin{tabular}{l l}
        Navn: & Petter S. Storvik \\
        Alder: & 22 år \\ 
        Linje: & Elektronikk \\
        Styrke: & Målrettet \\
        Svakhet: & Noe begrenset relevante kunnskaper
    \end{tabular}
\end{table}

I og med at studenter som har hatt EiT kun har ytret seg negativt om faget, var jeg tildels negativt innstilt i begynnelsen. 
Landsbyens tema var mitt førstevalg, og skulle være interessant nok til å kompensere for prosessdelen. 
Til tross for min negative holdning var jeg innstilt på å få mest mulig ut av EiT.
Derfor valgte jeg å være åpen mot prosessdelen og prøve å engasjere meg mest mulig i denne delen av faget. 
I og med at jeg ikke kjente noen der fra før, var jeg spent på å bli kjent med gruppen og begynne med det tekniske prosjektet. 

