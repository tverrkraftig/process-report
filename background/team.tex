\section{Team}
Til å begynne med må et team/arbeidsgruppe defineres. 
Det finnes mange definisjoner på dette, men gruppen valgte å rette seg etter Katzenbach og Smith som lyder som følger: 

\begin{center}
\textit{Et team defineres som flere personer som arbeider sammen for å oppnå et felles mål. \newline
Medlemmene må være avhengige av hverandre på en eller annen måte.}
\newline 
\citep{katzenbach}
\end{center}

En gruppe settes med andre ord ofte sammen for å oppnå et mål som er vanskelig/umulig å oppnå alene. 
Det er ikke slik at det alltid er fordelaktig å jobbe i gruppe, men komplekse flerfaglige/uoversiktlige oppgaver blir gjerne enklere å løse i grupper.
Gruppearbeid kan også gi løsninger preget av en større grav av nyskapning og kreativitet. 
Generelt er den største motivasjonen for å arbeide i en gruppe at større prosjekter har en veldig stor grad av tverrfaglighet. 
Dette er noe som kommer godt frem i arbeidslivet også \citep{levin}. 
En gruppe bør ikke bestå av mer enn 10 personer, men 5-6 regnes av Katzenbach som den optimale størrelsen.
