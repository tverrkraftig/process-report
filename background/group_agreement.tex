\section{Utarbeidelse av samarbeidsavtale}
En samarbeidsavtale mellom gruppemedlemmene blir sett på som en viktig og nødvendig del av et effektivt gruppearbeid. 
MRPI-modellen, som er basert på arbeid av David Kolb (1986) er en god mal for en samarbeidsavtale. 
Denne sier at en avtale bør bestå av:
\begin{itemize}
  \item \textbf{M}ål
  \item \textbf{R}oller
  \item \textbf{P}rosedyrer
  \item \textbf{I}nterpersonlige forhold
\end{itemize}
Med andre bør avtalen bør inneholde svar på hvilke mål gruppen har, de ulike rollene i gruppen, hvordan beslutninger skal tas, hvordan gruppemedlemmene skal forholde seg til hverandre og øvrige formaliteter \citep{levin}.

\subsection*{Første utgave}
Den første utgaven av avtalen ble utarbeidet i starten av gruppearbeidet. 
Avtalens utforming var basert på en mal som ble utdelt i landsbyen. 
Denne malen fortalte hvilke kategorier som kunne være lure å ha med, samt noen viktige punkter innen disse kategoriene. 
På grunn av at gruppens medlemmer ikke kjente hverandres måter og jobbe på var gruppen klar over at samarbeidsavtalen måtte revideres senere ut i gruppearbeidet. 
\vspace{\secspace}

Vi valgte å bruke følgende kategorier:
\begin{itemize}
    \item \textbf{Leveranse} - Omfatter arbeidmenge, metode, mål i faget, o.l.
    \item \textbf{Trivsel} - Det som går på det sosiale i gruppen. 
    \item \textbf{Læring} - Dette punktet følte gruppen var nødvendig for å maksimere læringsutbyttet i EiT. 
    \item \textbf{Annet} - Her står det som ikke passser inn i de andre kategoriene. Er også nevnt hvordan de ulike rollene i gruppen fordeles. 
\end{itemize}
Samarbeidsavtalen ligger som Vedlegg 1. 

\subsection*{Andre utgave}
Midt i prosjektoppgaven ble samarbeidsavtalen revidert. 
Dette ble gjort i forbindelse med et initiativ fra fasilitatorene, og ble tatt felles for hele landsbyen. 
Gruppen var enige om at en systematisk gjennomgang av hele samarbeidsavtalen der vi diskuterte hva som hadde fungert bra og hva som kunne vært bedre. 
Den reviderte samarbeidsavtalen ligger som Vedlegg 2. 
\vspace{\secspace}

\noindent Det som ble forandret eller lagt til var: 
\begin{enumerate}
  \item Å følge dagsplanen: Dagsplanen er et eget dokument som sier hvilke faste aktiviteter som skal gjøres og når. Leder har ansvar for at disse følges. 
  \item Alle aksjoner som bestemmes skal settes inn i eget aksjonsdokument. Dette dokumentet er for å lettere holde styr på det som blir vedtatt.
  \item Lederrollen skal rullere. Slik vil alle få best mulig læringsutbytte av EiT. 
\end{enumerate}

Den første samarbeidsavtalen var med andre ord ganske bra, derfor ble ikke så mye forandret/lagt til. 
Dette er også en indikator på at gruppen var flink til å følge det som ble satt i avtalen. 
