\section{Gruppens personligheter}
En gruppe består av flere enkeltindivider. 
Personligheten til alle i gruppen virker inn på hvordan gruppemedlemmene forholder seg til hverandre. 
Det kan være viktig å kartlegge hvilke personligheter som er i gruppen, slik at det blir lettere å forstå hvorfor gruppen utvikler seg som den gjør.

\subsection{Meyers-Briggs personlighetsindikator}
Meyers-Briggs er bassert på teorier av Carl Gustaf Jung (1921).
Målet er å klassifisere de psykologiske forskjellene mellom mersoner.
Bokstavene i fet skrift vil gi en person en beskrivelse på fire bokstaver, det vil si 16 mulige personlighetstyper.
Denne beskrivelsen vil da gjenspeile person best mulig, og kan derfor brukes til å beskrive gruppemedlemmer og deres forskjeller.
\vspace{\secspace}

\begin{table}[H]
    \centering
    \begin{tabular}{| l | l c r |}
        \hline
        \textbf{Holdning} & Utadvent(\textbf{E}xtraversion) & $\leftrightarrow$ & Innadvendt(\textbf{I}ntrovert) \\ \hline
        \textbf{Funksjon} & Sansing(\textbf{S}ensing) & $\leftrightarrow$ & Intuisjon(i\textbf{N}tuition) \\ \hline
        \textbf{Funksjon} & Tenking(\textbf{T}hinking) & $\leftrightarrow$ & Følelse(\textbf{F}eeling) \\ \hline
        \textbf{Livsstil} & Vurdering(\textbf{J}udging) & $\leftrightarrow$ & Oppfattelse(\textbf{P}ercieving) \\
        \hline
    \end{tabular}
    \label{tab:meyersbriggs}
    \caption{Meyers-Briggs akser}
\end{table}

Tabell~\ref{tab:meyersbriggs} viser hvilke forskjellige akser som Meyers-Briggs legger vekt på, samt hvilke bokstaver som benyttes for de forskjellige resultatene.
\textbf{Holdning} er ganske selvforklarende, og forteller om en person er utadvent eller innadvent. 
\textbf{Funksjon} deles i to, Sansing-Intuisjon og Tenking-Følelse. 
Sansing-Intuisjon går på hvordan man tolker og oppfatter ny informasjon. 
En person med preferanse for sansing vil stole på klar informasjon, mens intuisjon gjerne har bedre forståelse av abstrakte begreper. 
Tenking-Følelse dreier seg om å ta beslutninger. 
Tenking beskriver en person som tenker over beslutningen logisk og konsistent. 
En person som beskrives av følelser vil sette seg inn i situasjonen og identifisere seg med de involverte parter.
\textbf{Livsstil} beskriver hvordan en person møter verden. 
Vurdering vil da si at personen vurderer det som skjer mer enn å bare oppfatte det. 
Med dette menes at de tenker over hendelser og de ulike inntrykkene. 
En av svakheten til Meyers-Briggs er at hvordan de enkelte personene definerer de ulike begrepene vil variere ut fra hvilken personlighetstype de har. 
Derfor er ikke aksene så uavhengige som de egentlig skulle vært. 


\subsection{Gruppemedlemmene} \label{chap:meyersgroupmembers}
For å avgjøre hvilke personligheter gruppa bestod av brukte vi et gratis verktøy på internett \citep{mbtest}. 
Dette verktøyet bestemmer din personlighet på grunnlag av 70 spørsmål. 
Grunnen til at vi brukte dette verktøyet fremfor å klassifisere oss seg, var at et slikt verktøy resulterer gjerne i bedre resultater. 
Kvaliteten på akkurat denne testen vet vi forøvrig ikke noe om. 
Alle tok denne testen hver for seg og resultatet ble drøftet. 

\begin{table}[H]
    \centering
    \begin{tabular}{| l | l | l l l l |}
        \hline
        \textbf{Anders} & ENTJ & \textbf{E}xtravert(33\%) & i\textbf{N}tuitive(50\%) & \textbf{T}hinking(25\%) & \textbf{J}udging(67\%)  \\ \hline
        \textbf{Emil} & INTJ & \textbf{I}ntrovert(56\%) & i\textbf{N}tuitive(12\%) & \textbf{T}hinking(1\%) & \textbf{J}udging(11\%)  \\ \hline
        \textbf{Petter} & ISTJ & \textbf{I}ntrovert(30\%) & \textbf{S}ensing(40\%) & \textbf{T}hinking(75\%) & \textbf{J}udging(67\%)  \\ \hline
        \textbf{Odd} & ENTJ & \textbf{E}xtravert(11\%) & i\textbf{N}tuitive(38\%) & \textbf{T}hinking(25\%) & \textbf{J}udging(67\%) \\ \hline
        \textbf{Ole}  & ISTJ & \textbf{I}ntrovert(33\%) & \textbf{S}ensing(01\%) & \textbf{T}hinking(75\%) & \textbf{J}udging(01\%)  \\
        \hline
    \end{tabular}
    \label{tab:meyersmemb}
    \caption{Meyers-Briggs resultat}
\end{table}

Beskrivelsene av de forskjellige personlighetstypene er hentet fra \textit{Meyers-Briggs foundation} \citep{mbtypes}.
\vspace{\secspace}

\textbf{ENTJ} (the commander) tar gjerne avgjørelser og lederskap enkelt. 
De ser raskt ulogiske og ineffektive prosedyrer og liker langsiktig planlegging og målsetting. 
Vanligvis godt belest og kunnskapsrik. 
Liker å utvide kunnskapene sine og lære ting videre til andre. 
\vspace{\secspace}

\textbf{INTJ} (the mastermind) er ofte innsiktsfull om andre. 
De søker mening i og sammenhenger mellom forskjellige idéer og forhold, og er pliktoppfyllende og forpliktet til sine faste verdier. 
Ofte organisert og prøver å finne ut hvordan man kan tjene det felles beste. 

\textbf{ISTJ} (the inspector) er ofte stille og satser mye på pålitelighet og grundighet. 
De innehar ofte praktiske ferdigheter, og er ansvarlige og realistiske. 
Arbeider jevnt og trutt med det som skal gjøres, til tross for distraksjoner. 
Setter ofte pris på tradisjoner og liker at ting er velorganiserte og ryddige. 

\subsection{Sammensetningen}
Det kommer godt frem av Meyers-Briggs, og forsåvidt presentasjonen av gruppemedlemmene, at gruppen har i stor grad lik faglig sammensetning.
Ofte brukes to begreper om dette, de blir ifølge videoforelesninger\citep{video} i EiT: 
\vspace{\secspace}

\textbf{Homogen gruppe} vil si at gruppemedlemmene har liten faglig spredning. 
En slik gruppe vil være mest effektiv når prosjektet har begrenset omfang og er lite flerfaglig. 
Fordelen med en slik gruppe er at gruppemedlemmene kommer ofte godt overens og har derfor gjerne mindre sosiale utfordringer. 
Det er også noen bakdeler med en homogen gruppe. 
Homogene grupper har en tendens til å unngå risikoer, og dermed går glipp av muligheter. 
De har også problemer med å tilpasse seg dynamiske situasjoner.
\vspace{\secspace}

\textbf{Heterogen gruppe} er det motsatte av en homogen gruppe. 
Dette er en gruppe preget faglig spredning og er ofte bedre til å ta viktige beslutninger samt tenke nytt og kreativt enn homogene grupper. 
Til tross for dette er det ikke gitt at en faglig spredt gruppe resulterer i suksess. 
Ofte kan heterogene grupper ha større utfordringer når det kommer til intern kommunikasjon eller problemer med at medlemmer blir defensive og derfor mindre produktive. 
\vspace{\secspace}

Da gruppen er et godt eksempel på en homogen gruppe kan noen av utfordringene over bli reelle.
% TODO skriv en setning eller to mer på avslutningen av dette avsnittet
