\chapter*{Sammendrag}
%\section{Sammendrag}
Eksperter i Team er et fag som skal gi masterstudenter erfaring med, og innsikt i prosjektarbeid.
Denne rapporten beskriver hvordan gruppedynamikken vår har utviklet seg gjennom prosjektet i Eksperter i Team.
Vi skal se hvordan gruppemedlemmenes bevissthet i forhold til sin egen rolle i gruppa endrer seg, og hvordan de behersker å bytte på roller fra dag til dag.
Rapporten viser hvordan vi har reflektert over ting som påvirker samarbeidet, og iverksatt aksjoner for å forbedre det. 
Vi viser hvilke utgangspunkt gruppen hadde før prosjektet startet, og hvor vi endte opp til slutt. 
Gruppen som helhet sitter igjen med en mye større innsikt i en gruppes roller og dynamikk. 
Samtlige har tilegnet seg ''verktøy'' som kan benyttes i fremtidige samarbeidsprosjekter.
\vspace{\secspace}

Første del av samarbeidet var preget av en ``bli-kjent''-fase.
Denne delen var mindre produktiv, og det var her gruppen skulle forme sine mål og normer. 
Etterhvert som gruppen definerte arbeidsoppgavene ble arbeidet mer strukturert og effektiviteten økte. 
Samtidig som gruppens effektivitet økte, økte også samholdet innad i gruppen. 
Det ble gjennomført en rullerende lederstruktur som hjalp alle med å få bedre innsikt i sin egen rolle, samt utfordre seg selv til å prøve noe nytt. 
Dette hadde varierende hell med tanke på effektivitet, men økte læringsutbyttet betraktelig. 
\vspace{\secspace}

Kommunikasjonen i gruppa var i begynnelsen skjevt fordelt. 
Noen snakket mye, mens andre holdt seg mer i bakgrunnen. 
Dette ble mye bedre etter at gruppen hadde en åpen diskusjon om kommunikasjonsmønstret, og dermed ble alle mer bevisst på hvordan de selv ble oppfattet av de andre. 
Gruppen har opparbeidet seg et positivt kommunikasjonsklima som oppfordrer til åpne diskusjoner og lav terskel for å ta opp ting. 

\vfill
\begin{center}
Vi ønsker å takke landsbyleder Sven Fjeldaas for teknisk veiledning, og fasilitatorene Henrik André Pedersen og Hanne Schøld Sæterdal for hjelp med prosess-delen.
\end{center}
