\section{Kommunikasjonsteori}
Det er viktig med god kommunikasjon innad i gruppen for at samarbeidet skal fungere. 
Med kommunkasjon menes utveksling av meninger, oppfatninger, tanker og følelser som skjer mellom mennesker.
Å kommunisere er å gjøre et forsøk på skape en lignende sinnstilstand som man selv har, hos noen andre.
\vspace{\secspace}

Kommunkasjonsprosessen består alltid av minst to roller, en sender og en/flere mottakere. 
Senderen skal formidle noe til mottakeren og dette skal skje uten missforståelser. 
Dette forutsetter at senderen selv vet hva som skal formidles og han/hun må kunne sette det over i tale eller skriftelig kommunikasjon. 
Ofte er det store problemet i kommunikasjon selve forståelsen til mottakeren. 
For at dette skal gå best mulig må sender tilpasse budskapet til mottakerens kompetanse, og mottakeren må kunne stille spørsmål for å oppklare eventuelle uklarheter. 
Derfor er det viktig at sender og mottaker arbeider sammen om kommunikasjonsprosessen.
\vspace{\secspace}

At en gruppe har et positivt kommunikasjonsklima vil si at medlemmene føler seg emosjonelt komfortable. 
I slike grupper blir det ofte saklige og åpne diskusjoner der alle kommer med sine ideer og spørsmål. 
Negativt kommunikasjonsklima er preget av defensiv adferd. 
Ofte kan medlemmer få støtende respons når de kommuniserer, som for eksempel å bli møtt med sarkasme eller negativ evaluering, og dette resulterer i mer og mer defensiv adferd. 
Vi sier derfor at et negativt kommunikasjonsklima er selvforsterkende. 
\vspace{\secspace}

For å forbedre kommunikasjonsferdighetene er det viktig å:
Tenke over hvordan man stiller spørsmål, lytte aktivt, gi konstruktiv tilbakemelding og vise evne til å takle følelser. 
Hvilke spørsmål som bidrar til positiv kommunikasjon kan være så mangt. 
Unngå enkle ja/nei spørsmål og heller still spørsmål som oppmuntrer til diskusjon, da det ofte er bedre at gruppen diskuterer ulemper og fordel fremfor å si hvem som er enige eller uenige. 
Aktiv lytting vil si å stille oppklarende spørsmål underveis og gjøre slik at den som snakker føles seg hørt og forstått. 
Konstruktive tilbakemeldinger kan være både positive og negative. 
Det er viktige at den som får tilbakemeldingen ikke får inntrykk av at det blir gitt kritikk, men heller at det opplyses om forbedringspotensiale.
