\section{Gruppeutvikling}

Rolfsen \& Lenin fokuserer på følgende modeller:
\vspace{\secspace}
%TODO: need some libtex references up in this, yo

\textbf{Åtte faser}, utviklet av Rosen (1987), er en modell som består av åtte forskjellige faser som gruppen går gjennom.
Gruppen utvikler seg i de fire første fasene. Her øker også ytelsen etterhvert som gruppen tar form.
Etter disse fasene kommer to faser med veldig intensivt arbeid og høy ytelse, som følges av en stagnasjonsfase. 
Siste fase er en avslutningsfase eller en fornyelsesfase.
Denne modellen antar ikke en gitt rekkefølge på fasene, men tar høyde for at gruppen kan hoppe frem og tilbake mellom de ulike fasene iløpet av gruppens levetid. 
Svakheten er at \textit{åtte faser} passer best for grupper som går over lengre tid, og da ikke kan benyttes i alle tilfeller. 
\vspace{\secspace}
