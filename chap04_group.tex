\chapter{Gruppens utvikling}

For å beskrive gruppens utvikling gjennom hele prosjektet brukes modellene som ble beskrevet i teori-delen. 
Gruppen mener at halvtidsmodellen blir for liten for å beskrive utviklingen. 

\section{Gruppens utvikling}

\begin{figure}
    \begin{tikzpicture}[remember picture, overlay, xshift=-0.5cm, yshift=14.3cm]
        \foreach \x in {0,-1.5,-3,-4.5,-6,-7.5,-9,-10.5,-12,-13.5}%,-15,-16.5}
            \draw [color=white, fill=LightGray, blur shadow] (0,\x) -- (0.5,\x+0.5) -- (0.5,\x-1) -- (0,\x-1.5) -- (-0.5,\x-1) -- (-0.5,\x+0.5) -- (0,\x);
        \draw [color=white, top color=red, bottom color=DarkRed, blur shadow] (0,0) -- (1,1) -- (1,-2) -- (0,-3) -- (-1,-2) -- (-1,1) -- (0,0);
        \draw (0,-0.5) node[anchor=north, color=white] {\Large \textbf{Fase 1}};
        \draw [color=white, top color=red, bottom color=DarkRed, blur shadow] (0,0-9) -- (1,1-9) -- (1,-2-9) -- (0,-3-9) -- (-1,-2-9) -- (-1,1-9) -- (0,0-9);
        \draw (0,-0.5-9) node[anchor=north, color=white] {\Large \textbf{Fase 2}};
        %\draw (0,0) -- (0,-20);
    \end{tikzpicture}
\end{figure}

\begin{adjustwidth}{1cm}{0cm}
\lipsum[0-4]
\end{adjustwidth}


%%% Local Variables: 
%%% mode: latex
%%% TeX-master: "report"
%%% End: 
