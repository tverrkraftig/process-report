\section{Emil}
''Vanligvis er jeg ganske stille i diskusjoner i starten av gruppeprosjekter, og EiT var ikke noe unntak.
Jeg fant relativt raskt et område jeg kunne mye om som var essensielt for prosjektet og kunne dermed lett bidra til den endelige utformingen av prosjektet.
I starten så jeg på grupperefleksjonene mest som en måte å få grunnlag til å skrive en god prosessrapport og dermed få en god karakter.
Et vendepunkt for min del var etter å ha sett den første videoforelesningen, hvor det ble tatt opp ting som kunne være destruktive for gruppearbeid som jeg kjente veldig godt igjen fra tidligere gruppeprosjekter jeg har jobbet med.
Etter dette ble jeg mer interessert i det å selv erfare forskjellige faktorer som spiller inn på et godt samarbeid for å kunne gjøre det bedre i fremtiden.
Jeg har i løpet av EiT blitt flinkere til å kommunisere med resten av gruppen mer regelmessig.
Mer konkret er det de små tingene jeg har blitt flinkere til å kommunisere, enten det er bekreftelser eller små frustrasjoner over sidespor og distraksjoner.'' 
\hfill - Emil
\vspace{\secspace}

Emil var en mann av få ord.
Helt fra begynnelsen har gruppen fått god hjelp av Emil. 
På grunn av at oppgaven i veldig stor grad dreide seg om programmering, og dette er noe Emil er flink til, har han alltid vært vår ''wikipedia''. 
Utover prosjektet ble han den som sa ifra dersom diskusjonen gikk treigt. 
Som leder var han veldig flink til å følge de aksjonene som var skrevet opp, og til å holde tidene (pauser, o.l). 
Når han jobbet med sine arbeidsoppgaver var han ofte svært målrettet og produktiv. 
En ting hele gruppen opplevde med Emil var at han var flink til å sette seg inn i mottakerens situasjon når han forklarte noe. 
Derfor måtte man ofte ikke spørre så mange forklarende spørsmål, da Emil selv forsto hva han måtte utdype. 