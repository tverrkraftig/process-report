\section{Odd}
``Jeg er en veldig avslappet person.
Jeg er selv i stand til å ta initiativ, men synes ofte det er enklere å gli inn bak andres mens jeg gir støtte og utfordrer.
Ettersom jeg måtte gjennom EiT, som ville si at jeg måtte bruke tiden min på det, tenkte jeg det ville være like greit å gjøre så lite motstand som mulig.
Onsdager ble bare en dag med forhåndsbestemt opplegg, dager jeg ikke kunne bruke til \textit{andre ting}.
Mye av tiden min går med til prokrastinering -- dette har jeg jobbet med å finne ut av i løpet av det siste halvåret -- og finner ofte inspirasjon eller motivasjon mens jeg holder på med noe annet.
Da jeg innså at det kun var en person i gruppen jeg ikke hadde første- eller annengrads kjennskap til fra før, og at vi alle sammen var fra samme fakultet med tilhørighet til to linjeforeninger totalt følte jeg meg snytt av den reklamerte EiT-opplevelsen om tverrfaglighet og fremmede.
Det føltes ut som jeg produserte veldig lite konkret på prosjektet.
Jeg hadde som regel unnskyldninger til hvorfor jeg ikke hadde gjort noe i løpet av en landsbydag.
Jeg la etterhvert merke til at jeg er veldig flink til å lage en forklarende fortelling som får handlingene og valgene mine til å virke fornuftige og logiske i ettertid.
I løpet av semesteret har jeg innsett at jeg kan dikte opp forklaringer for flere av mulighetene jeg må velge mellom, og at det virker som om det ikke er noe særlig sammenheng mellom hvor gode forklaringene virker og hvor fornuftig valget faktisk var.
At jeg gang på gang prokrastinerte vekk eller var overraskende lite produktiv i løpet av landsbydagene tror jeg hjalp meg på vei til denne innsikten.''
\hfill - Odd
\vspace{\secspace}
