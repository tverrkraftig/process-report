\section{Anders Strand}
''Jeg starter dette prosjektet med mye iver og entusiasme. 
Dette medførte at jeg i stor grad satte rammene for det tekniske prosjektet, som i stor grad ble formet av mine forkunnskaper.
Jeg gikk i en rolle jeg er veldig kjent med, nemlig lederrollen. 
Det er en rolle som har gitt meg gode resultater tidligere, både i forbindelse med frivillige verv og andre skole-prosjekter.
I løpet av Eit-opplegget ble det lagt godt til rette for at vi kunne reflektere over våre roller i gruppen.
Særlig ga en videoforelesning meg innsikt i de ubevisste årsakene til hvorfor jeg likte å ta på meg lederrollen og sette rammer. 
videoforelesninger forklarte hvordan mange reagerer etter "fight or flight"-prinsippet når man havner i en ny og ukjent situasjon.
Enkelte velger å trekke seg tilbake å bli passive, og på den måten skjerme seg. 
Andre velger den stikk motsatte løsningen, nemlig å ta kontroll og sette premissene for situasjonen. 
Jeg tilhører den siste kategorien, og handlet på omtrent nøyaktig samme måte som teorien tilsier.
Utover Eit-prosessen har jeg bevisst jobbet med å la andre komme til.
Jeg har latt være å hoppe på spørsmål med en gang, men vente å se om andre fyller hullet jeg da ikke lenger fyller. Det har vært et lærerikt eksperiment som har gjort meg mer bevisst på de forskjellige rollene i en gruppe.''
\vspace{\secspace}

I begynnelsen var Anders fremtredende og tok mye initiativ i planleggingsfasen. 
Etter bevisstgjøringen på den ulikes rolle i gruppen var Anders en av de som ble beskrevet som frempå og hadde lett for å ta lederrollen. 
Denne bevisstgjøringen resulterte i en rullerende lederrolle, og dette gjorde også at Anders ikke kunne ta styringen i den grad han var vant med. 
Han taklet denne overgangen bra. 
Det var noen ganger tendenser til at han prøvde og ta styringen igjen, oftest når lederen var litt sløv, men han var flink til å hente seg inn igjen. 
Anders har også blitt flinkere til å være mer rett fram og ikke pakke tilbakemeldinger for godt inn. 
Dette har hatt en positiv effekt på gruppens kommunikasjonsklima. 

% TODO skrive dette avsnittet bedre??? 