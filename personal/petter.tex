\section{Petter}
''I begynnelsen av prosjektet var jeg veldig bevisst på det jeg tenkte på som mine begrensede fagkunnskaper. 
Det vil si at jeg var i liten grad med på å forme prosjektet, men synes det hørtes spennende ut å lære noe nytt om temaet. 
På grunn av at jeg hadde lite å komme med ble jeg oppfattet som stille. 
Selv var jeg bevisst på at jeg lett tar en mer passiv rolle. 
Dette stemte forøvrig godt overens med Meyers-Briggs personlighetstest. 
På grunn av at dette ble tatt opp veldig tidlig ble jeg veldig bevisst på at gruppen oppfattet meg slik. 
Derfor prøvde jeg å delta mer i diskusjoner, selv om jeg nødvendigvis ikke hadde noe faglig direkte knyttet opp mot det som ble diskutert. 
Da lederrollen begynte å rullere fikk jeg også prøvd meg som leder, dette med litt varierende hell. 
Grunnen til det var nok at hva som krevdes av lederen varierte med hvilke obligatoriske aktiviteter som var planlagt for dagen. 
Generelt mener jeg at jeg har blitt flinkere til å delta aktivt i diskusjoner innad i gruppen. 
Jeg har også blitt flinkere til å se hvordan jeg oppfattes av andre og hva som påvirker dette inntrykket. 
Alt i alt er jeg fornøyd med det jeg fikk ut av EiT. '' \hfill - Petter
\vspace{\secspace}

Petter var stille i begynnelsen av semesteret.
Det kunne gå en hel lunsj uten at han sa noe.
Det virket som han alltid hadde gjort noe hjemme.
Etterhvert åpnet han seg for gruppen og deltok oftere i samtaler og diskusjoner, selv de mindre seriøse.
Til tross for at gruppen oppfattet han som noe stille i begynnelsen av prosjektet taklet han lederrollen.
Petter holdt blikket rettet på rapportene fra starten av og fylte hullet skapt av de andre gruppenes laber interesse for dette aspektet ved prosjektet og tok ansvar for dette.
Dette ga de andre gruppemedlemmene spillerom til å fokusere på andre deler av prosjektet og utfylte gruppens egenskaper.
