\chapter{Log \& gruppens utvikling}
For å dokumentere gruppens fremgang ble det skrevet en logg/refleksjon etter hver landsbydag. 
Denne loggen var i begynnelsen litt lett, men etterhvert ble gruppen flinkere til å skrive gode grupperefleksjoner. 
Her følger sammendrag av grupperefleksjonene som ble skrevet gjennom hele EiT, samt en analyse på hvor gruppen er i forhold til teorier/modeller. 

\section{1. landsbydag (15.01.2014)}
Den første landsbydagen var preget av mange \textit{bli-kjent aktiviteter}. 
Det var først mot slutten av dagen at gruppen satt samlet og diskuterte sine tanker rundt EiT og hvilken vei vi skulle dreie prosjektet.
Til tross for at dette var første gang gruppen var samlet, var det ikke noen som ble overkjørt av andre. 
Gruppen preges av at noen er mer stille og innesluttet enn andre, mens andre har lettere for å ta styringen når ting skal diskuteres. 
Alle er ganske samstemte når det kommer til å legge bort sine initielle tanker om EiT og møte prosessdelen med åpent sinn. 
\vspace{\secspace}
Analyse/Teori

\section{2. landsbydag (22.01.2014)}
Sammendrag
\vspace{\secspace}
Analyse/Teori

\section{3. landsbydag (29.01.2014)}
Sammendrag
\vspace{\secspace}
Analyse/Teori

\section{4. landsbydag (05.02.2014)}
Sammendrag
\vspace{\secspace}
Analyse/Teori

\section{5. landsbydag (12.02.2014)}
Sammendrag
\vspace{\secspace}
Analyse/Teori

\section{6. landsbydag (19.02.2014)}
Sammendrag
\vspace{\secspace}
Analyse/Teori

\section{7. landsbydag (26.02.2014)}
Sammendrag
\vspace{\secspace}
Analyse/Teori

\section{8. landsbydag (05.03.2014)}
Sammendrag
\vspace{\secspace}
Analyse/Teori

\section{9. landsbydag (12.03.2014)}
Sammendrag
\vspace{\secspace}
Analyse/Teori

\section{10. landsbydag (19.03.2014)}
Sammendrag
\vspace{\secspace}
Analyse/Teori

\section{11. landsbydag (26.03.2014)}
Sammendrag
\vspace{\secspace}
Analyse/Teori

\section{12. landsbydag (dd.mm.2014)}
Sammendrag
\vspace{\secspace}
Analyse/Teori

\section{13. landsbydag (dd.mm.2014)}
Sammendrag
\vspace{\secspace}
Analyse/Teori
